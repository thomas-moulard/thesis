\documentclass[a4paper,twoside,openright]{book}

\usepackage[T1]{fontenc}
\usepackage[utf8x]{inputenc}

\usepackage{lmodern}

\usepackage[francais]{babel}

\usepackage{array}
\usepackage{mathtools}
\usepackage{amssymb}
\usepackage{amsthm}
\usepackage{mathtools}
\usepackage{makeidx}
\usepackage{graphicx}
\usepackage{url}
\usepackage{epigraph}
\usepackage{hyperref}
\usepackage{natbib}

\usepackage{fancyhdr}
\pagestyle{fancy}

\usepackage{hyperref}

% Setup pdf meta-data.
\hypersetup{ % Modifiez la valeur des champs suivants
pdfauthor = {Thomas Moulard},
pdftitle = {TBD},
pdfsubject = {TBD},
pdfkeywords = {Mots-clefs},
pdfcreator = {PDFLaTeX},
pdfproducer = {PDFLaTeX}
}


% Please fancyhdr.
\setlength{\headheight}{25pt}

% Customize chapter head.
\makeatletter
\renewcommand\chapter{\if@openright\cleardoublepage\else\clearpage\fi
                    \thispagestyle{empty}%
                    \global\@topnum\z@
                    \@afterindentfalse
                    \secdef\@chapter\@schapter}
\def\@makechapterhead#1{%
  \vspace*{50\p@}%
  {\parindent \z@ \raggedright \normalfont
    \ifnum \c@secnumdepth >\m@ne
      \if@mainmatter
        \huge\bfseries\thechapter\enspace
      \fi
    \fi
    \Huge \bfseries #1\par\nobreak
    \vskip 40\p@
  }}
\makeatother

% Add definition definition.
\newtheorem{mydef}{Définition}
\newtheorem{myexample}{Exemple}

% Enforce math style.
\everymath{\displaystyle}

% Generate index.
\makeindex

\begin{document}
\bibliographystyle{plainnat}
%
\author{Thomas Moulard}
\title{TBD}
\date{July 2012}
\maketitle
%
\frontmatter
\tableofcontents{\thispagestyle{fancy}}
%
\mainmatter 
\chapter{Introduction}
\label{chap:intro}

\epigraph{Text text text text text text text text text text text text
  text text text text text text text text text text text text text
  text text text text text text text text text text text text text
  text text text text text text text text text text text text text
  text text text text}{Some Author}
\clearpage

\section{Introduction}

\chapter[Optimisation]{Optimisation et généricité}
\label{chap:roboptim}

\epigraph{Text text text text text text text text text text text text
  text text text text text text text text text text text text text
  text text text text text text text text text text text text text
  text text text text text text text text text text text text text
  text text text text}{Some Author}
\clearpage

\section[Optimisation]{Optimisation numérique et robotique}

\subsection{Introduction à l'optimisation numérique}

La robotique s'appuie sur différents outils mathématiques pour
résoudre les problèmes auxquels elle se trouve confrontée. Parmi ces
outils, l'optimisation numérique figure à une place de choix.


L'optimisation numérique est une branche des mathématiques permettant
de choisir automatiquement la meilleure solution parmi un ensemble de
solutions possibles. Les applications possibles couvrent des domaines
variés de la recherche opérationnelle aux statistiques et bien sûr la
robotique. Dans ce domaine particulier, l'optimisation numérique
permets par exemple d'optimiser un mouvement pour en améliorer
certains caractéristiques ou bien encore trouver la prochaine commande
à envoyer aux moteurs du robot afin de réaliser une tâche donnée.



Résoudre un problème d'optimisation numérique signifie trouver les
meilleurs paramètres solutionnant le problème. Par meilleurs, il faut
entendre les paramètres minimisant la valeur d'une fonction de
coût. Intuitivement, on peut se représenter cette fonction comme un
indicateur de la ``qualité'' de la solution trouvée. Une solution
présentant un coût fort sera donc peu satisfaisante tandis qu'une
solution présentant un coût faible sera, elle, très attrayante. Tout
le raisonement est ensuite fondé sur la linéarité -- au moins locale!
-- de cette fonction de coût. Plus simplement, près d'une mauvaise
solution, il n'y aura que des solutions légèrement meilleures et
légèrement pires, mais jamais trop différentes. De ce fait, il
``suffit'' de chercher dans quelle direction aller pour se diriger
vers les meilleurs solutions pour finir par les trouver. Cette
approche implique une limitation majeure: si la fonction de coût n'est
pas convexe, on ne trouvera pas forcément la meilleure solution
globale -- parmi toutes les solutions -- mais ma meilleure solution
locale. C'est à dire qu'au voisinage de cette solution, toutes les
autres solutions sont de moins bonnes qualités. Dans cette situation,
tous les solveurs arrêtent leur recherche, mais rien ne garantit
qu'ailleurs il n'existe pas une autre solution de meilleure qualité.


Ce raisonnement permets de trouver les meilleurs paramètres pour un
problème donné, mais il est rare qu'une fonction de coût seule puisse
capturer complètement un problème donné. En fonctionnant de manière
indépendante du problème sous-jacent, on confère aux solveurs une
grande polyvalence. Mais par là même, il devient nécessaire
d'expliciter sous quelles conditions une solution au problème donné
est acceptable. Prenons un exemple: pour aller d'un point $A$ à un
point $B$ le plus rapidement possible, se déplacer avec une vitesse
infinie est sans aucun doute la solution optimale. Cependant, une
telle réponse ne présente guère d'intérêt dans la mesure où aucun
système physique ne sera en mesure de l'executer.


Deux stratégies peuvent alors être choisies. La première est
d'exprimer directement dans la fonction de coût cette information
supplémentaire. Dans le cadre de l'exemple précédemment cité, on
pourrait concevoir une fonction de coût réalisant la somme du temps de
déplacement et de la vitesse moyenne du robot. De ce fait, une vitesse
trop grande pénalise le résultat et par ce biais, le solveur tentera
plutôt de transformer la trajectoire géométrique plutôt que de jouer
sur la vitesse de déplacement du système. La limite de cette approche
est simple à comprendre: au fur et à mesure que les biais s'accumulent
-- le plus souvent en sommant les uns aux autres --, le coût initial
peut se retrouver ignoré par le solveur car numériquement
négligeable. La fonction de coût n'a alors plus aucune justification
physique et ses variations peuvent devenir de plus en plus difficiles
à analyser pour le solveur.  Une seconde solution consiste à ajouter
des contraintes au problème. Ces contraintes sont exprimées sous la
forme d'équation ou d'inéquation déterminant si une solution donnée
est une solution acceptable. En robotique, des contraintes très
courantes sont les bornes sur les butées articulaires. Il est rare sur
un robot que les axes puissent bouger de manière totalement libre. De
ce fait, il est courant d'ajouter des contraintes inégalités
spécifiant que tel ou tel valeur de joint doit rester dans un
intervalle donné.


En exprimant comment résoudre un problème pratique via la construction
d'une fonction de coût $f$ et de contraintes $g_i > 0$,
\textbf{modélise} le problème afin qu'il soit soluble par optimisation
numérique. Ce procédé n'est pas trivial car la qualité de la
modélisation impacte lourdement à la fois les temps de calcul et la
qualité du résultat final.



\subsection{Modélisation mathématique d'un problème}


Résoudre un problème d'optimisation revient à trouver une solution pour:

\begin{equation}
  \min_{\mathbf{x} \in \mathbb{R}^n} f(x) \text{ sous la contrainte } \mathbf{x} \in \mathbf{X}
\end{equation}

où $f : \mathbb{R}^n \mapsto \mathbb{R}$ est la fonction de coût et
$\mathbf{X} \subset \mathbb{R}^n$ est l'espace des solutions
admissibles. Cet espace est habituellement défini par un ensemble de
contraintes égalités et inégalités:

\begin{equation}
  \mathbf{X} \equiv \left\{ 
  \begin{array}{l l}
    c_i (x) = 0    & \quad i \in \xi \\
    c_j (x) \leq 0 & \quad j \in \nu \\
  \end{array} \right.
\end{equation}

$c_i$, $c_j$ sont respectivement les ensembles de contraintes égalités
et inégalités. $i$ et $j$ des indices identifiant les contraintes.


L'optimisation numérique a pour objectif de développer des stratégies
-- algorithmes -- pouvant déterminer $\mathbf{x} \in \mathbf{X}$
minimisant les valeurs de la fonction $f$.


En absence d'argument fort tel que la convexité de $f$ ainsi que des
contraintes $c_i, i \in \xi \cup \nu$, la solution $\mathbf{x}$
fournie par le solveur peut être un minimum local.


En effet, la plupart des méthodes d'optimisation tentent de trouver un
minimum en raffinant itérativement une solution à partir d'un candidat
initial $\mathbf{x_{\text{init}}} \in \mathbf{X}$. Le solveur, dès
lors, nécessite un critère d'arrêt à ce processus itératif. Ce critère
est donné par les conditions de Karush-Kuhn-Tucker ou conditions KKT.

\begin{mydef}\label{def:chap1_kkt}
Si $\mathbf{x} \in \mathbf{X}$ est un minimum local, alors il existe
$\lambda_i, i \in \xi$ et $\mu_j, j \in \nu$ des constantes non-nulles
appelées multiplicateurs KKT tels que:
%
\begin{description}
\item[\textbf{Critère de stabilité}] \begin{equation}
  \nabla f(x) + \sum_{i \in \xi} \lambda_i c_i(x) + \sum_{j \in \nu} \mu_i c_i(x) = 0
\end{equation}
\item[\textbf{Critère de faisabilité primale}] \begin{equation}
\left\{ 
\begin{array}{l l}
  c_i (x) = 0    & \quad i \in \xi \\
  c_j (x) \leq 0 & \quad j \in \nu \\
\end{array} \right.
\end{equation}
\item[\textbf{Critère de faisabilité duale}] \begin{equation}
\mu_j \geq 0, j \in \nu
\end{equation}
\item[\textbf{Critère de complémentarité}] \begin{equation}
\mu_j c_j(x) = 0, j \in \nu
\end{equation}
\end{description}
\end{mydef}

Il est à noter que la Définition~\ref{def:chap1_kkt} fournit des
conditions suffisantes afin de déterminer si un point est un minimum
local. Sous conditions de régularité, et dans le cas où le problème
est convexe, les conditions KKT sont nécessaires et suffisantes.


\subsection{Zoologie des solveurs}

test


\section[FIXME]{De la dif\-ficulté à passer d'un paradigme uni\-que à une implémentation unifiée}

test

\subsection{Solveurs disponibles}

test

\subsection{Motivation}

test


\section{Architecture logicielle de Roboptim}

test

\section{Scénarii d'utilisation}

test

\subsection[FIXME]{\'Ecriture d'une application robotique complexe: optimiser la trajectoire de marche pour un robot humanoïde}

test

\subsection{Intégration d'un solveur tiers et extensibilité de RobOptim}

test

\section{Discussions}

test

\section{Conclusion}

test

\chapter{Suivi de trajectoire}
\label{chap:suivi}

\epigraph{Text text text text text text text text text text text text
  text text text text text text text text text text text text text
  text text text text text text text text text text text text text
  text text text text text text text text text text text text text
  text text text text}{Some Author}
\clearpage

\section{Problématique}
\section{\'Etat de l'Art}
\section{Prospectives}

\chapter{Primitives de mouvement}
\label{chap:primitive}

\epigraph{Text text text text text text text text text text text text
  text text text text text text text text text text text text text
  text text text text text text text text text text text text text
  text text text text text text text text text text text text text
  text text text text}{Some Author}
\clearpage

\section{Problématique}

Le chapitre précédent a introduit la possibilité d'asservir une
trajectoire de marche sur un robot humanoïde. Cependant, les tâches
accomplies par les robots humanoïdes ne se limitent pas à la
locomotion et il est intéressant de se demander s'il est possible de
combiner aux tâches de navigation d'autres tâches asservies par les
données capteurs afin de réaliser des comportements complexes. De
manière indirecte, la question qui se pose ici est celle de la limite
à placer entre d'une part raisonnement numérique et d'autre part
raisonnement logique. Ce problème récurrent de la robotique et pour
lequel il n'existe pas, de consensus au sein de la communauté trouve
ici une solution élégante. En effet, un contrôleur fondé sur le
paradigme de la pile de tâche ne se limite pas à une description
d'objectifs ou de contraintes robotiques dans des espaces plus
naturels que l'espace des configurations ou l'espace cartésien, il
ouvre surtout la voie à des méchanismes de supervision décidant à quel
moment insérer telle ou telle tâche à un niveau de priorité donné. La
pile de tâche réalise donc la jonction entre d'une part, le monde du
calcul numérique puisque la finalité du système est de calculer la
commande du système et d'autre part le monde de la logique dans lequel
les utilisateurs humains souhaitent exprimer leur \emph{desiderata} au
système robotique: va jusqu'à la cuisine, apportes moi la bouteille,
ouvres la porte, etc. Il est clair que ce type d'ordre nécessite la
résolution d'abstractions et que la logique mathématique semble être
un moyen particulièrement adapté pour y parvenir. Ces méchanismes de
décision de haut niveau sont appelés ``superviseurs'' et ont pour
objectif d'instantier et d'orchestrer tous les composants d'une
architecture robotique. Dans le cadre de notre architecture, il faut
donc pouvoir instantier des piles de tâches à partir d'une description
de haut niveau des ordres passés au robot. Ce chapitre propose un
langage de description des tâches permettant de faire le lien entre
représentation logique et représentation numérique.


Nous allons commencer par détailler l'état de l'Art et en particulier
les autres travaux aillant trait aux architectures robotiques haut
niveau et à l'ordonnancement de tâches ainsi que plus généralement aux
applications robotiques complexes. Dans un second temps, le langage de
description sera décrit et quelques scénarii types seront
démontrés. Enfin, nous nous concentrerons sur les problèmes
d'asservissement posés par les robots humanoïdes avant de conclure.


\section{\'Etat de l'Art}

FIXME

\section{Description d'un mouvement robotique complexe}


Décrire le comportement d'une pile de tâche revient à décrire deux
éléments primordiaux: d'une part les tâches exprimées dans le solveur
et de l'autre les différents changements d'état de la pile au cours du
mouvement. Concernant les tâches, il s'agit ici de représenter des
fonctions mathématiques génériques ne présentant pas de point commun
permettant une représentation générique de ces dernières. Par exemple
si l'on se limitait aux fonctions linéaires $\mathbf{A}(t) \mathbf{x}
+ \mathbf{b}(t)$, encoder la matrice $\mathbf{A}$ et le vecteur
$\mathbf{b}$ serait envisageable pour un ensemble discret de valeur
$t$ données. Le cas évoqué est trop contraint pour notre problème et
les modélisations informatiques développées dans le chapitre 1 n'aide
pas: elles ont pour but de venir fournir un modèle pour l'expression
d'une fonction mathématique sous la forme d'un algorithme et non pas
sous la forme d'une donnée structurée pouvant facilement être encodée.
Le choix a donc été fait de définir un ensemble de tâches permettant
de réaliser un certain nombre d'actions intéressantes. Rien n'empêche
alors cet ensemble de ``primitives'' d'être étendu au cours du temps
mais il nécessite l'extension du modèle de description défini ici.


La second partie est la représentation des changements d'état. Un
changement d'état du solveur survient quand une tâche est: ajoutée,
supprimée ou bien encore quand sa priorité est modifiée. De manière
générale les transitions peuvent être soit temporelles -- avance de 5
mètres puis saisit la poignée de la porte --, soit logique -- si tu es
à moins de 10 cm de la position finale, arrêtes-toi --. Pour
comprendre la stratégie choisit, il faut garder à l'esprit que pour
réaliser un scénario complexe certains comportements seront intégrés
dans la boucle de temps réel tandis que d'autres -- typiquement les
informations capteurs -- seront traitées à l'extérieur. Les deux
catégories d'événements, temporelles ou logiques, ne mettent pas en
jeu les mêmes méchanismes. Les événéments temporels sont décidés à
l'avance et doivent être exécutés avec une grande précision pour
assurer un comportement correct du système tandis que les événéments
logiques sont le résultat d'un méchanisme de décision externe.

De ce fait, la stratégie adoptée a été de permettre la représentation
de transitions temporelles dans le modèle de description uniquement
tandis que l'on considère que les transitions événémentielles, de fait
plus lentes et difficiles à encoder, sont gérées à l'extérieur du
contrôleur par un logiciel décisionnel aillant la possibilité de
regénérer le mouvement s'il devient invalide suite à la réception
d'une nouvelle donnée capteur. De nombreux logiciels adaptés à cette
tâche ont été développés tel que
SMACH\footnote{\url{http://www.ros.org/wiki/smach}}.


\subsection{Primitive de mouvement}

La première tâche a été de définir des primitives de mouvement de
haut-niveau. Les primitives proposées sont:
\begin{enumerate}
\item La locomotion: mouvement synchronisé des deux jambes et du
  centre de masse du robot pour réaliser un déplacement tout en
  assurant sa stabilité.
\item La manipulation et le contact: mouvement réalisant le placement
  d'une partie spécifiée du robot à un emplacement donné dans l'espace
  Euclidien. La tâche peut contraindre la position et/ou la rotation
  du corps à positionner.
\item Regard ou asservissement visuel: le robot doit maintenir un
  point dans son champ de vision.
\end{enumerate}

FIXME tache bas niveau featurepoint6d, etc. def math d'une primitive, où mentionner l'entrée de la correction?

\subsection{Primitive de locomotion}

Une tâche de locomotion est définie intialement comme une série de
points de contact à réaliser pour atteindre une position
finale. Chacun des points de contact étant annoté par l'effecteur
devant réaliser le contact à cet endroit. À partir de ces
informations, une trajectoire des effecteurs réalisant les appui est
déduite ainsi que du centre de masse pour préserver l'équilibre
dynamique du système. Afin de pouvoir utiliser les techniques décrites
dans le chapitre FIXME, nous nous limiterons à la marche sur un sol
plat utilisant le pied gauche et le pied droit du robot. Le calcul de
la trajectoire des effecteurs et du centre de masse est encore un
problème ouvert à l'heure actuelle. Les modèles simples peuvent être
embarqués dans les contrôleurs au prix de nombreuses simplifications
détaillées dans le chapitre précédent tandis que les modèles les plus
compliqués nécessitent une résolution hors du contrôleurs rendant le
comportement du système non-réactif. Le chapitre précédent a proposé
une stratégie pour fusionner les avantages des deux approches. Nous
considérons ici la totalité de l'approche développée au cours du
chapitre précédent comme une implémentation d'une primitive de
locomotion. En particulier, l'erreur de positionnement du robot est
considérée comme une entrée de la primitive de mouvement.

\subsection{Primitive de manipulation}



\subsection{Langage de description}

\subsection{Asservissement sur un capteur et fermeture de la boucle}

\section{Scénarii de mouvements}
\subsection{Locomotion simple}

\begin{figure}
  \begin{center}
\begin{verbatim}
duration: 200 # durée complète du mouvement (secondes)

# Éléments de mouvement
motion:
  # Primitive de locomotion
  - walk:
      interval: [0, 200] # Date de début et de fin de la primitive

      # Pile de pas
      footsteps:
      - {x: 0.15, y: -0.19, theta: 0., slide1: 0., slide2: -0.76}
      - {x: 0.15, y: +0.19, theta: 0.1, slide1: -1.00, slide2: -0.76}
      # etc.
\end{verbatim}
  \end{center}
  \caption{Plan de mouvement pour une séquence de marche non-asservie.}
\end{figure}

\subsection{Locomotion asservie}

\begin{figure}
  \begin{center}
\begin{verbatim}
duration: 200 # durée complète du mouvement (secondes)

# correction maximum autorisée sur un pas (x, y en mètre, theta en radians)
maximum-correction-per-step: {x: 0.02, y: 0.02, theta: 0.05}

# Éléments de mouvement
motion:
  # Primitive de locomotion
  - walk:
      interval: [0, 200] # Date de début et de fin de la primitive

      # Pile de pas
      footsteps:
      - {x: 0.15, y: -0.19, theta: 0., slide1: 0., slide2: -0.76}
      - {x: 0.15, y: +0.19, theta: 0.1, slide1: -1.00, slide2: -0.76}
      # etc.

# Asservissement des tâches
control:
  # Asservissement via le système de capture de mouvements.
  - mocap:
      weight: 1.
      tracked-body: left-ankle
      perceived-body: left-foot
\end{verbatim}
  \end{center}
  \caption{Plan de mouvement pour une séquence de marche asservie sur
    un système de localisation -- ici un système de capture de
    mouvement --.}
\end{figure}


\subsection{Scénario complexe avec tâche d'atteinte et asservissement corps complet}

\section{De la difficulté à localiser un robot humanoïde}
\subsection{Forces et limites du robot humanoïde HRP-2}
\subsection{Vision pour la robotique humanoïde}
\paragraph{Difficultés classiques de la vision}
\paragraph{Stabilisation de la tête}
\paragraph{Pourquoi un robot n'est pas une caméra volante}
\paragraph{Résultats de la localisation utilisant la vision sur le robot humanoïde HRP-2}

\section{Résultats}

\section{Conclusion}

\chapter{Applications robotiques complexes}
\label{chap:integration}

\epigraph{Text text text text text text text text text text text text
  text text text text text text text text text text text text text
  text text text text text text text text text text text text text
  text text text text text text text text text text text text text
  text text text text}{Some Author}
\clearpage

\section{Architecture}
\section{Simulation transparente}
\section{Modélisation unifiée d'un système robotique}
\section{Applications robotiques complexes}

\chapter{Conclusion}
\label{chap:conclusion}

\epigraph{Text text text text text text text text text text text text
  text text text text text text text text text text text text text
  text text text text text text text text text text text text text
  text text text text text text text text text text text text text
  text text text text}{Some Author}
\clearpage

\section{Conclusion}

%
%
\backmatter
%
% Bibliography
\nocite{*}
\bibliography{thesis}
%
% Index
\cleardoublepage
\addcontentsline{toc}{chapter}{Index}
\printindex
%
\end{document}
